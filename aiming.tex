\documentclass{article}
\usepackage[hidelinks]{hyperref}
\usepackage{mathtools}
\usepackage{nicefrac}


\title{Aiming Mechanics in World of Tanks}
\author{CrySpy}

\begin{document}
\maketitle
\tableofcontents
\pagebreak
\section{Introduction}
Within the game of World of Tanks a very important aspect is how long one has to wait
before one can fire. This is presented to the player as a shrinking reticle, commingnly known as aiming circle.
The size of the aiming circle is deterined by movement of the tank and firing of the gun.
The paremeters the game displays regarding this is dispersion, also known as accuracy,
and aiming time. Aiming time espeically is confusing as it might mislead the player to believe
it is a measure of how long it takes to fully aim. 

\section{Actual aiming time}
\subsection{Derviving the aiming function}

%wfactor is not 4/10 it's e to the negative -> ~0.36
\paragraph{}First of all we must understand how the aiming circle decreases in size. 
To do this we must look into what aiming time actually means, and how it affects aiming.
The game says aiming is a measure of how long it takes from the current aiming circle
to decrease by 60\% or to 40\%, of its current size, that is \(\frac{4}{10}f(t)=f(t+t_{\mathrm{aiming}})\)
The wiki lists this as the time it takes to shrink to a \(\nicefrac{1}{3}\) of its size.
This is troublesome as both of these are incorrect. Data mining the game files we can find the factor
to be \(\nicefrac{1}{e}\) which is roughly equal to 36\%.

\paragraph{}To derive the actual function we function we can look at a simplfied example.
We assume that the size of the aiming circle is 1 to start with. Then we look at the size of the aimicircle
for every aiming time interval. Then we get \[1, \frac{1}{e}, \left(\frac{1}{e}\right)^2, \left(\frac{1}{e}\right)^3 \cdots\]
If we want to use a different starting size, such as 2.78 we get
\[2.78, 2.78\times\frac{1}{e}, 2.78\times\left(\frac{1}{e}\right)^2, 2.78\times\left(\frac{1}{e}\right)^3 \cdots\]
As demonstrated the fuction is mererly to mutliply by \(\nicefrac{1}{e}\) for every aiming time interval.
With that in mind we can create a function where the exponent of \(\nicefrac{1}{e}\) is an integer for every multiple of the aiming time
and we get 
\[f\left(t\right) = \mathrm{Dispersion_0} \times \left(\frac{1}{e}\right)^{\frac{t}{t_{\mathrm{aiming}}}}\]
Simplying
\begin{equation}
    f\left(t\right) = \mathrm{Dispersion_0} \times e^{-\frac{t}{t_{\mathrm{aiming}}}}
    \label{eq:aimfunc}
\end{equation}
\subsection{Solving the aiming function for t}
Now that we have a function that describes how the aiming circle shrinks over time
it is temping to see if it can be used to dervive a formula for the time it takes to aim.
If we take equation \eqref{eq:aimfunc} and replace \(f(t)\)  with the final dispersion value we get something we can solve for t
\[\mathrm{Dispersion_{final}} = \mathrm{Dispersion_0} \times e^{-\frac{t}{t_{\mathrm{aiming}}}}\]
\[\ln \left(\frac{\mathrm{Dispersion_{final}}}{\mathrm{Dispersion_0}} \right) = \ln\left( e^{-\frac{t}{t_{\mathrm{aiming}}}}\right)\]
\[\ln \left(\frac{\mathrm{Dispersion_{final}}}{\mathrm{Dispersion_0}}\right) = -\frac{t}{t_{\mathrm{aiming}}} \times \ln e\]
\[\ln \left(\frac{\mathrm{Dispersion_{final}}}{\mathrm{Dispersion_0}}\right) = -\frac{t}{t_{\mathrm{aiming}}}\]
\[\ln \left(\frac{\mathrm{Dispersion_{final}}}{\mathrm{Dispersion_0}} \right) \times -t_{\mathrm{aiming}} = t\]
\begin{equation}
    \label{eq:aimtime}
    t = - \ln \left(\frac{\mathrm{Dispersion_{final}}}{\mathrm{Dispersion_0}} \right) \times t_{\mathrm{aiming}}
\end{equation}
With equation \eqref{eq:aimtime} we can easily determine the time it takes to aim to fully aim a gun, or to any other point we'd like without the need
to plot equation \eqref{eq:aimfunc}. At this point we are have the knowledge to study how a gun aims and determine how long it takes, but 
we do not still know how to determine how much the aiming circle grows when the tank moves.

\section{Talking about bloom}
The formula for determining how the aiming circle grows relative to the tank's movement and other factors
is present in the game files as presented here.
D = disperion, S = speed, mv = movement, ta tr = tank traverse, tu tr = turret traverse
\begin{equation}
    \label{eq:bloom}
    D = D_{f} \times \sqrt{1 + \left(D_{mv}\times S_{mv}\right)^2 + \left(D_{ta tr}\times S_{ta tr}\right)^2
+ \left(D_{tu tr}\times S_{tu tr}\right)^2 + \mathrm{shot}^2}
\end{equation}

There are a few things to note: the final dispersion, or accuracy, of the tank scales the growth of the aiming circle.
The speed of any sort of movement is multiplied by a coefficent. This means that a slower moving tank will have a  smaller aiming circle than a faster moving tank if all else is equal.


\section{A further look into the functions}
\subsection{Aiming function}
We can also see that \(\mathrm{Disperson_{final}}\) is both present in equation \eqref{eq:aimtime} and \eqref{eq:bloom}.
This means that the accuracy of a tank does not matter in determining the time it takes to fully aim.
With this we can modify equation \eqref{eq:aimtime} slightly and we get 
\[t = - \ln \left(\frac{1}{\mathrm{Dispersion_{mod}}} \right) \times t_{\mathrm{aiming}}\]
\begin{equation} \label{eq:gen}
    t = \ln \left(\mathrm{Dispersion_{mod}}\right) \times t_{\mathrm{aiming}}
\end{equation}

Where \[D_{\mathrm{mod}} = \sqrt{1 + \left(D_{mv}\times S_{mv}\right)^2 + \left(D_{ta tr}\times S_{ta tr}\right)^2
+ \left(D_{tu tr}\times S_{tu tr}\right)^2 + \mathrm{shot}^2}\]

While this simplifies the expression it is not so much of use by itself. While it does show that
the accuracy of a tank does not determine how long it takes to fully aim, it says nothing about how big
the aiming circle is at any point as that is removed from the formula. What we gain is 
a generalization of the formula
\subsection{The effect of equipment}
What comes to mind is how equipment affects the time needed to (fully) aim. As previously shown,
accuracy does not matter in this regard, and thus IAU is not useful in this regard. It does aim in faster
to the base accuracy, however. What we will look at is how GLD, VSTABS, and IRM factor in.
\paragraph{GLD}
is the easiest to analyze. GLD will decrease the aiming time by a factor of \(\nicefrac{1}{1.1}\)
When looking at \eqref{eq:gen} we observe that \(t_{\mathrm{aimng}}\) is just a scaling factor of \(t\)
Thus we can conclude that the time to aim reduced by GLD is the same as the reduction in aiming time, namely
the factor of \(\nicefrac{1}{1.1}\) or approximately 91\%
\paragraph{VSTABS}
and IRM are very similar. They work by reducing the dispersion penalties on tank movement,
tank traverse, turret/gun traverse, and after shooting. IRM gives a lesser reduction than VSTABS while 
also adding a bonus increase to the traverse speeds. Both are equal, \(\pm 10\%\). Let us take a closer look at this by
looking at how it affects the penalty from tank traverse
\[\left(D_{tatr}\times S_{tatr}\right)^2\]
If we apply the bonuses as \(\left(1 \pm B\right)\) where \(B = 0.1\) we get
\[\left( \left(1 - B\right) D_{tatr}\times (\left(1 + B\right) S_{tatr}\right)^2\]
If we re arrange things we get
\[\left( \left(1 - B\right) \left(1 + B \right) D_{tatr}\times S_{tatr}\right)^2 \]
We contract
\[\left( \left(1 - B^2 \right) D_{tatr}\times S_{tatr} \right)^2\]
We observe that the bonus applied is \( \left(1 - B^2 \right)\), if we evaluate it for 10\% and 12.5\% we get
\[ \left(1 - 0.1^2 \right) = 0.99\]
\[ \left(1 - 0.125^2 \right) \approx 0.984\]
Thus we can conclude that IRM does not provide a substantional bonus to the dipserion while traversing.
\paragraph{In general}
we can look at how the bonus from VSTABS and IRM applies to time to aim.
First we recall \eqref{eq:bloom} and apply the bonus B to all of the terms.
\[D_{\mathrm{mod}} = \sqrt{1 + \left( B \times D_{mv}\times S_{mv}\right)^2 + \left(B \times D_{ta tr}\times S_{ta tr}\right)^2
+ \left(B \times D_{tu tr}\times S_{tu tr}\right)^2 +  \left( B \times \mathrm{shot}\right)^2}\]
If we factor out B
\[D_{\mathrm{mod}} = \sqrt{1 + B^2 \left( \left( D_{mv}\times S_{mv}\right)^2 + \left( D_{ta tr}\times S_{ta tr}\right)^2
+ \left( D_{tu tr} \times S_{tu tr} \right)^2 + \mathrm{shot}^2 \right) } \]
We can observe that if the addition of 1 was not present then B would reduce the modified dispersion as a factor
If we simplify the expression so that the dispersion and speed factors are added and treated as a single variable we get
\[D_{\mathrm{mod}} = \sqrt{1 + B ^2 \times D_p}\]
If we divide the modified dispersion with a bonus active by a modified dispersion and take the limit of it as \(D_p\) grows we get
\[ \lim\limits_{D_p \to \infty}  \frac{\sqrt{1 + B ^2 \times D_p}}{\sqrt{1 + D_p}} \] 
We can combine the square roots
\[ \lim\limits_{D_p \to \infty} \sqrt{\frac{1 + B ^2 \times D_p}{1 + D_p}} \]
Since this is a infinity over infinity limit we need to do something about that.
We can divide every term by \(D_p\) and we get
\[ \lim\limits_{D_p \to \infty} \sqrt{ \frac{\nicefrac{1}{D_p} + \frac{B^2\times D_p}{D_p}}{\nicefrac{1}{D_p} + \nicefrac{D_p}{D_p}} }\]
Simplify
\[ \lim\limits_{D_p \to \infty} \sqrt{ \frac{0 + B^2}{0 + 1}}\]
\[ \lim\limits_{D_p \to \infty} \sqrt{B^2} = B\]
And thus we can conclude that the max impact VTSABS (and by extension IRM) on the dispersion is the bonus
as an factor, but only as the penalties grows large. In other words VSTABS will reduce the aim circle by 20 \% 
as a theoretical max, and this will occur while driving or traversing fast. 
\paragraph{This}
is however not the direct impact on time to aim. Recall \eqref{eq:gen}. We can now add the factor B in to look at the theoretical max
reduction in time to aim.
\[ t = \ln \left(\mathrm{B \times Dispersion_{mod}}\right) \times t_{\mathrm{aiming}}\]
We can split the logarithm

\[ t = \left(\ln\left(\mathrm{Dispersion_{mod}}\right) + \ln \left( B\right)\right) \times t_{\mathrm{aiming}} \]
Note that \(\ln B\) will be negative since \( B < 1 \)
Expanding

\[t = \ln \left( \mathrm{Dispersion_{mod}} \right) \times t_{\mathrm{aiming}} + \ln \left( B \right) \times t_{\mathrm{aiming}} \]

We can see that B will contribute to a flat decrease in time to aim if look at the theoretical max.
This is however not a static value

\section{GLD vs VSTABS}
Now it is time to look closer at which point GLD becomes better than both IRM and VSTABS

\[ \ln \left(\mathrm{Dispersion_{modB}}\right) \times t_{\mathrm{aiming}} = \ln \left(\mathrm{Dispersion_{mod}}\right) \times t_{\mathrm{aiming}} \times \mathrm{GLD}\]
We see that aiming time does not matter, so we can remove it
\[ \ln \left(\mathrm{Dispersion_{modB}}\right) = \ln \left(\mathrm{Dispersion_{mod}}\right) \times \mathrm{GLD}\]
Unforunately solving this equation symbolically is impossilbe, however we can solve it numerically.
GLD intersects IRM at approximately 2.47 and VSTABS at approximately 11.32. The former is quite easily surpassed while the latter is not.









% t_B - t_VS/GLD <= t_turrB - t_turrIRM
% t_B - t_VS/GLD <= angle/S_turrB - angle/S_turrIRM
% t_B - t_VS/GLD <= angle/S_turrB - angle/(1.1*S_turrB)
% t_B - t_VS/GLD <= (1.1*angle)/(1.1*S_turrB) - angle/(1.1*S_turrB)
% t_B - t_VS/GLD <= (1.1*angle - angle)/S_turrIRM
% t_B - t_VS/GLD <= (angle(1.1-1))/S_turrIRM
% t_B - t_VS/GLD <= (angle(0.1))/S_turrIRM


% irm vs
% ln(D_mod)*t_a - ln(D_modB)*t_a
% t_a * ln(D_mod/D_modB) <= (angle(0.1))/S_turrIRM

% irm gld
% ln(D_mod)*t_a - ln(D_mod)*t_a*0.909
% t_a(ln(D_mod) - ln(D_mod)*0.909)
% t_a(ln(D_mod)(1-0.909))
% t_a(0.091*ln(D_mod))


% 0.091*t_a*ln(D_mod) <= (angle(0.1))/S_turrIRM		


\end{document}